    % !TEX TS-program = pdflatexmk

\documentclass[wideaddress]{vitae}

%\renewcommand*{\bibliographyhead}[1]{}

\usepackage[letterpaper]{geometry}
\usepackage{marvosym}


\usepackage{url}

%\usepackage[labeled,resetlabels]{multibib}
%\usepackage[resetlabels]{multibib}
\usepackage[maxbibnames=99,backend=biber,style=numeric-verb,sorting=ydnt]{biblatex}




\usepackage{enumitem}
\usepackage{etaremune}
\usepackage{textcomp}
\usepackage{tabto}
\usepackage{datetime}


% %Define bibliographies
% \newcites{B,V,J,C,O,PO,N,P,T,PA}%
%    {Book Chapters:,%
%     Edited Volumes:,%
%     Refereed Journal Articles:,%
%     Refereed Conference Proceedings:,%
%     Refereed Workshops Papers:,%
%     {Refereed Posters, Extended Abstracts and Abstracts:},
%     Preprints and Other Non-refereed Publications:,%
%     Panels:,%
%     Presentations and Talks:,%
%     }

% \ifx true false

% -------------------------------------------------------------------------------------
% references in reverse order
% -------------------------------------------------------------------------------------

% Define a custom bibliography environment with modified left margin indentation
\defbibenvironment{bibliography}
  {\list
     {\printtext[labelnumberwidth]{%
        \printfield{labelprefix}%
        \printfield{labelnumber}}}
     {\setlength{\labelwidth}{0.7cm} % Adjust the value as needed
      \setlength{\leftmargin}{0cm}   % Adjust the value as needed
      \addtolength{\leftmargin}{\labelwidth}%
      \addtolength{\leftmargin}{\labelsep}%
      \setlength{\itemsep}{\bibitemsep}%
      \setlength{\parsep}{\bibparsep}%
      \renewcommand*{\makelabel}[1]{\hss##1}}}
  {\endlist}
  {\item}

% Count total number of entries in each refsection
\AtDataInput{%
  \csnumgdef{entrycount:\therefsection}{%
    \csuse{entrycount:\therefsection}+1}}

% Print the labelnumber as the total number of entries in the
% current refsection, minus the actual labelnumber, plus one
\DeclareFieldFormat{labelnumber}{\mkbibdesc{#1}}
\newrobustcmd*{\mkbibdesc}[1]{%
  % \number\numexpr\csuse{entrycount:\therefsection}+1-#1\relax}
  \number\numexpr\csuse{entrycount:\therefsection}+1-#1.\relax}
% \fi

% \DeclareFieldFormat{labelnumber}{#1.}
\DeclareFieldFormat{labelnumberwidth}{#1}
% -------------------------------------------------------------------------------------
% -------------------------------------------------------------------------------------


\ifx true false
\addbibresource[label=B]{master-bibliography.bib}
\addbibresource[label=V]{master-bibliography.bib}
\addbibresource[label=J]{master-bibliography.bib}
\addbibresource[label=C]{master-bibliography.bib}
\addbibresource[label=O]{master-bibliography.bib}
\addbibresource[label=PO]{master-bibliography.bib}
\addbibresource[label=N]{master-bibliography.bib}
\addbibresource[label=P]{master-bibliography.bib}
\addbibresource[label=T]{master-bibliography.bib}
\addbibresource[label=PA]{master-bibliography.bib}
\fi
\addbibresource{master-bibliography.bib}


%\renewcommand{\labelitemi}{$\diamond$}
\newcommand{\VT}{Virginia Tech}

%%%%%%%%%%%%%%%%%%%%%%%%%%%%%%%%%%%%%%%%%%%%%%%%%%%%%%%%%
% New commands and environments

% This defines how the name looks
\newcommand{\bigname}[1]{
	\begin{center}\fontfamily{phv}\selectfont\Huge\scshape#1\end{center}
}

% A ressection is a main section (<H1>Section</H1>)
\newenvironment{ressection}[1]{
	\vspace{4pt}
	{\fontfamily{phv}\selectfont\Large#1}
	\begin{itemize}
	\vspace{3pt}
}{
	\end{itemize}
}

%mm/yy format
\newdateformat{monthyeardate}{%
  \monthname[\THEMONTH], \THEYEAR}


\makeatletter
\def\@biblabel#1{#1. }

\renewenvironment{thebibliography}[1]
     {\subsection*{\refname}%
      \@mkboth{\MakeUppercase\refname}{\MakeUppercase\refname}%
      \list{\@biblabel{\@arabic\c@enumiv}}%
           {\settowidth\labelwidth{\@biblabel{ |||#1}}%
            \leftmargin\labelwidth
            \advance\leftmargin\labelsep
            \@openbib@code
            \usecounter{enumiv}%
            \let\p@enumiv\@empty
            \renewcommand\theenumiv{\@arabic\c@enumiv}}%
      \sloppy
      \clubpenalty4000
      \@clubpenalty \clubpenalty
      \widowpenalty4000%
      \sfcode`\.\@m}
     {\def\@noitemerr
       {\@latex@warning{Empty `thebibliography' environment}}%
      \endlist}
\makeatother

% %make my name bold in publications
% \def\FormatName#1{%
%   \def\myname{Regis Kopper}%
%   \edef\name{#1}%
%   \ifx\name\myname
%     \textbf{#1}%
%   \else
%     #1%
%   \fi
% }



%make my name bold in publications
\newtoggle{abx@bool@moreRegis}
\newtoggle{abx@bool@moreKopper}

\def\mkbibnamegiven#1{%
  \def\mygivenname{Regis}%
  \edef\name{#1}%
  \iftoggle{abx@bool@moreKopper}{\togglefalse{abx@bool@moreKopper}}{}%
  \ifx\name\mygivenname
    \iftoggle{abx@bool@moreRegis}{}{\toggletrue{abx@bool@moreKopper}}%
    \textbf{#1}%
  \else
    #1%
  \fi
}

\def\mkbibnamefamily#1{%
  \def\myfamilyname{Kopper}%
  \edef\name{#1}%
  \iftoggle{abx@bool@moreRegis}{\togglefalse{abx@bool@moreRegis}}{}%
  \ifx\name\myfamilyname
    \iftoggle{abx@bool@moreKopper}{}{\toggletrue{abx@bool@moreKopper}}%
    \textbf{#1}%
  \else
    #1%
  \fi
}


% A resitem is a simple list element in a ressection (first level)
\newcommand{\resitem}[1]{
	\item \begin{flushleft} #1 \end{flushleft}
}

% A ressubitem is a simple list element in anything but a ressection (second level)
\newcommand{\ressubitem}[1]{
	\item \begin{flushleft} #1 \end{flushleft}
}

% A resbigitem is a complex list element for stuff like jobs and education:
%  Arg 1: Name of company or university
%  Arg 2: Location
%  Arg 3: Title and/or date range
\newcommand{\resbigitem}[3]{
	\item
	\textbf{#1}---#2 \\
	\textit{#3}
}

% This is a list that comes with a resbigitem
\newenvironment{ressubsec}[3]{
	\resbigitem{#1}{#2}{#3}
	\begin{itemize}
}{
	\end{itemize}
}

% This is a simple sublist
\newenvironment{reslist}[1]{
	\resitem{\textbf{#1}}
	\begin{itemize}
}{
	\end{itemize}
}

% This is a simple numbered sublist
\newenvironment{resnumberedlist}[1]{
	\resitem{\textbf{#1}}
	\begin{enumerate}
}{
	\end{enumerate}
}

\author{Regis Kopper Publications}

\address{\url{https://orcid.org/0000-0003-2081-7061}\\
  Updated: \monthyeardate\today}


\begin{document}

\maketitle

\section{Publications}

\begin{refsection}
  \textbf{Book Chapters:}
  \nocite{Bowman:2019aa,McMahan:2014qe,Kopper:2003fk}
  \printbibliography[heading=none]
\end{refsection}

\begin{refsection}
  \textbf{Edited Volumes:}
  \nocite{Kopper:2021aa,Argelaguet:2020aa,Bebis:2015nr,Bebis:2015zr,TREVISAN2022A12}
  \printbibliography[heading=none]
\end{refsection}

\begin{refsection}
  \textbf{Refereed Journal Articles:}
    \nocite{enahora2024assessment,mcguirt2024multi,Souza:2021aa,Cao:2021ue,Wang:2021wy,liu2021psychophysiological,Palmeira:2020aa,Olk:2018aa,Rao:2018aa,Lercari:2017aa,Mehta:2017aa,Ragan:2015lq,Johnson:2014fk,Bacim2013,Robb:2013uq,Ragan:2013qf,Chuah:2013fr,Johnson2012,Figueroa:2010fk,Kopper2010,Ragan:2010kx}
  \printbibliography[heading=none]
\end{refsection}

\begin{refsection}
  \textbf{Refereed Conference Proceedings:}
  \nocite{Moreira:2023aa,Cao:2023aa,Palmeira:2021vp,Souza:2020aa,Souza:2018aa,Clements:2018aa,Cao:2018aa,Naz:2017aa,Zielinski:2017aa,Zielinski:2016by,Zielinski:2015kl,Rivera-Gutierrez:2014fk,Zielinski:2013fk,Chuah2012,Krishnan:2012dq,Bowman:2012ys,Kopper:2011zr,Bowman:2009fk,Rieder:2006nx,Kopper:2006,Kopper:2004fx,Grandi:2023ab}  \printbibliography[heading=none]
\end{refsection}

\begin{refsection}
  \textbf{Refereed Workshops Papers:}
  \nocite{Lofca:2023aa,Taweel:2023aa,Negrao:2023aa,Serras:2023aa,Grandi:2021aa,Bennett:2014aa,Lynch:2014yg,Zielinski:2014aa,Kopper:2012fk,Kopper:2011kx,Kopper:2000kl}
  \printbibliography[heading=none]
\end{refsection}

\nopagebreak[4]
\begin{refsection}
  \textbf{Refereed Posters, Extended Abstracts and Abstracts:}
  \nocite{kopperFirstModulAROpenSourceModular2025,okamura2024xr,lofca2022studying,okamura2024xr,mcguirt2023rural,Grandi:2023aa,Grandi:2020aa,Grandi:2019aa,Liu:2019aa,Zhao:2018aa,Appelbaum:2018aa,Appelbaum:2017aa,Borba:2017aa,Borba:2017ab,Correa:2017aa,Cutler:2017aa,Soares:2017aa,Zielinski:2017ab,Appelbaum:2016uk,Borba:2016aa,Borba:2016ab,Mehta:2016aa,Rao:2016xi,Soares:2016aa,Zielinski:2016fj,Burkins:2015fp,Olk:2015yu,Izatt:2014dk,Lercari:2014sp,Lercari:2014db,Zielinski:2014bh,Nicolelis:2014cr,Lercari:2013dl,Stinson:2011tg}
  \printbibliography[heading=none]
\end{refsection}

\begin{refsection}
  \textbf{Refereed Courses:}
  \nocite{jorge2023approaches}
  \printbibliography[heading=none]
\end{refsection}

\begin{refsection}
  \textbf{Preprints and Other Non-refereed Publications:}
  \nocite{Kopper:2009ij,Judge:2008vn,Kopper:2008dp}
  \printbibliography[heading=none]
\end{refsection}

\section{Patents}
\begin{refsection}
  \nocite{Kopper:2019af, Kopper:2020aa}
  \printbibliography[heading=none]
\end{refsection}

\end{document}